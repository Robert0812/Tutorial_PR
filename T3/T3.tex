\documentclass[compress,blue]{beamer}
\usepackage[latin1]{inputenc}
\usepackage{tikz}
\usepackage{mathtools}

\renewcommand\mathfamilydefault{\rmdefault}


\usetikzlibrary{shapes.arrows}
\tikzset{
    myarrow/.style={
        draw,
        fill=red,
        single arrow,
        minimum height=3.5ex,
        single arrow head extend=1ex
    }
}
\newcommand{\arrowup}{%
\tikz [baseline=-0.5ex]{\node [myarrow,rotate=90] {};}
}
\newcommand{\arrowdown}{%
\tikz [baseline=-1ex]{\node [myarrow,rotate=-90] {};}
}
\DeclarePairedDelimiter{\ceil}{\lceil}{\rceil}
\DeclarePairedDelimiter\floor{\lfloor}{\rfloor}
\newcommand{\argmin}{\operatornamewithlimits{argmin}}
\newcommand{\argmax}{\operatornamewithlimits{argmax}}
\newcommand{\bx}{\mathbf{x}}
\newcommand{\bX}{\mathbf{X}}
\newcommand{\bw}{\mathbf{w}}
\newcommand{\bS}{\mathbf{S}}

\usetheme{Warsaw}

\title[ENGG 5202 Pattern Recogntion Tutorial 3]{Tutorial 3: Dimensionality Reduction}
\author{Rui Zhao}
\institute{rzhao@ee.cuhk.edu.hk}
\date{Feb. 6, 2014}

\begin{document}

\begin{frame}
\titlepage
\end{frame}

\setbeamertemplate{enumerate items}[square]
\setbeamertemplate{itemize items}[square]

\begin{frame}{Outline}
\setbeamercovered{transparent}
	\begin{enumerate}
		\item<1-2> Lagrange Optimization
		\vspace{0.1in}
		\item<1> Generalized Rayleigh Quotient 
		\vspace{0.1in}
		\item<1> Exercises 
	\end{enumerate}
\end{frame}

\begin{frame}{1. Lagrange Optimization}
	Constrained optimization problem:
	\begin{align}
		\max_{\bx} ~&~ f(\bx), \\
		s.t. ~&~ g(\bx) = 0.
	\end{align}
	The solution can often be found by Lagrangian method. The Lagrangian is defined as:
	\begin{align}
		L(\bx, \lambda) = f(\bx) + \lambda g(\bx).
	\end{align}
\end{frame}

\begin{frame}{1. Lagrange Optimization}
	\textbf{Lagrangian Sufficiency Theorem}: Suppose there exist $\bx^* \in \bX$ and $\lambda^*$, such that $\bx^*$ maximize $L(\bx, \lambda^*)$ over all $\bx \in \bX$, and $g(\bx^*) = 0$. Then $\bx^*$ solves the optimization problem. \\
	\vspace{0.1in}
	\textbf{Proof}. 
	\begin{align}
		& \max_{\bx\in\bX, ~g(\bx) = 0} ~f(\bx)\\
		=& \max_{\bx\in\bX, ~g(\bx) = 0} ~[f(\bx) + \lambda^* g(\bx)] \\
		\leq &~~~~ \max_{\bx\in\bX} ~[f(\bx) + \lambda^* g(\bx)] \\
		= & ~~~~ ~[f(\bx^*) + \lambda^* g(\bx^*)] \\
		= & ~~~~ ~ f(\bx^*)
	\end{align} 
\end{frame}

\begin{frame}{1. Lagrange Optimization}
	\textbf{Solve Lagrange Optimization}: solve the unconstrained problem by taking the derivative w.r.t. $\bx$ and $\lambda$:
	\begin{align}
		&\frac{\partial L(\bx, \lambda)}{\partial \bx} = \frac{\partial f(\bx)}{\partial \bx} + \lambda\frac{\partial g(\bx)}{\partial \bx} = 0 \\
		&\frac{\partial L(\bx, \lambda)}{\partial \lambda} = 0 \\
		&g(\bx) = 0
	\end{align}
\end{frame}

\begin{frame}{Outline}
\setbeamercovered{transparent}
	\begin{enumerate}
		\item<1-2> Lagrange Optimization
		\vspace{0.1in}
		\item<2> Generalized Rayleigh Quotient 
		\vspace{0.1in}
		\item<0> Exercises 
	\end{enumerate}
\end{frame}

\begin{frame}{2. Generalized Rayleigh Quotient }
	\begin{block}{Fisher Criterion}
		\begin{align}
			J(\bw) = \frac{\bw^t  \bS_B \bw}{\bw^t \bS_W \bw}
		\end{align}
		$J(\bw)$ is the generalized Rayleigh quotient. A vector $\bw$ that maximizes $J(\cdot)$ must satisfy 
		\begin{align}
			\bS_B\bw = \lambda\bS_W\bw
		\end{align}
		for some constant $\lambda$.
	\end{block}	
\end{frame}

\begin{frame}{2. Generalized Rayleigh Quotient}
	Maximizing $J(\bw)$ is equivalent to 
	\begin{align}
		\max_{\bw} ~~& \bw^t \bS_B \bw \\
		s.t. ~~~& \bw^t \bS_W \bw = K
	\end{align}
	which can be solved using Lagrange multipliers.
\end{frame}

\begin{frame}{2. Generalized Rayleigh Quotient}
	Define the 	Lagrangian:
	\begin{align}
		L = \bw^t\bS_B\bw - \lambda (\bw^t \bS_W \bw - K)
	\end{align} 
	Maximize with respect to $\bw$:
	\begin{align}
		\nabla_{\bw}L = 2(\bS_B - \lambda\bS_W)\bw = 0
	\end{align}
	To obtain the solution:
	\begin{align}
		\bS_B\bw = \lambda \bS_W\bw
	\end{align}
\end{frame}

\begin{frame}{Outline}
\setbeamercovered{transparent}
	\begin{enumerate}
		\item<1-2> Lagrange Optimization
		\vspace{0.1in}
		\item<1-2> Generalized Rayleigh Quotient 
		\vspace{0.1in}
		\item<2> Exercises 
	\end{enumerate}
\end{frame}


\end{document} 